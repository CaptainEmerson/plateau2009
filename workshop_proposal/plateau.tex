\documentclass{sigplanconf}

\usepackage{ifthen}
\usepackage{colortbl}
\usepackage{tabularx}

\newboolean{hidecomments}
\setboolean{hidecomments}{true}
\ifthenelse{\boolean{hidecomments}}
{\newcommand{\nb}[2]{}}
{\newcommand{\nb}[2]{
    \fbox{\bfseries\sffamily\scriptsize#1}
    {\sf\small$\blacktriangleright$
      {#2} $\blacktriangleleft$}}}
\newcommand\apb[1]{\nb{APB}{#1}}
\newcommand\emh[1]{\nb{EMH}{#1}}


\begin{document}

\title{Evaluation and Usability of Programming Languages and Tools (PLATEAU)}

\authorinfo{Emerson Murphy-Hill}
           {University of British Columbia}
           {emhill@cs.ubc.ca}
\authorinfo{Shane Markstrum}
           {Bucknell University}
           {shane.markstrum@bucknell.edu}
\authorinfo{Craig Anslow}
           {Victoria University of Wellington}
           {craig@ecs.vuw.ac.nz}

\date{}

\maketitle
\begin{abstract}

  Programming languages exist to enable programmers to develop
  software effectively.  But how \emph{efficiently} programmers can
  write software depends on the usability of the languages and tools
  that they develop with.  The aim of this workshop is to discuss
  methods, metrics and techniques for evaluating the usability of
  languages and language tools.  The supposed benefits of such
  languages and tools cover a large space, including making programs
  easier to read, write, and maintain; allowing programmers to write
  more flexible and powerful programs; and restricting programs to
  make them more safe and secure. We plan to gather the intersection
  of researchers in the programming language, programming tool, and
  human-computer interaction communities to share their research and
  discuss the future of evaluation and usability of programming
  languages and tools. We are also interested in the input of other
  members of the programming research community working on related
  areas, such as refactoring, design patterns, program analysis,
  program comprehension, software visualization, end-user programming,
  and other programming language paradigms.

\end{abstract}

\category{D.3.0}{Programming Languages}{Standards}\\
\category{H.1.2}{User/Machine Systems}{Human Factors}


\terms{Human Factors, Languages}

\keywords{Evaluation, Programming Languages, Tools, Usability} 

\section{Main Themes and Goals}

At the Programming Languages Grand Challenges panel at POPL 2009, Greg
Morrisett claimed that one of the great neglected areas in programming
languages research is the bridge between programming languages and
human-computer interaction: the evaluation of the usability of
programming languages and tools. This is evident by the recent
research programs of major languages conferences such as POPL, PLDI,
OOPSLA, and ECOOP. The object-oriented conferences tend to have at
least one or two papers in the areas of corpus analysis or evaluation
methodologies, but the authors of the papers seem to avoid using the
results of their studies to make conclusions about the languages or
tools themselves. Software engineering and human-computer interaction
conferences tend to have better support of language usability analysis
(CHI 2009 had three tracks that showcase research in this direction),
but have limited visibility to the programming languages community.

Following on to our previous workshop of the same name at OOPSLA 2009
\footnote{http://ecs.victoria.ac.nz/Events/PLATEAU/},
this workshop aims to begin filling that void by
developing and stimulating discussion of usability and evaluation of
programming languages and tools with respect to language design and
related areas. We will consider: empirical studies of programming
languages; methodologies and philosophies behind language and tool
evaluation; software design metrics and their relations to the
underlying language; user studies of language features and software
engineering tools; visual techniques for understanding programming
languages; critical comparisons of programming paradigms, such as
object-oriented vs. functional; and tools to support evaluating
programming languages. We have two goals:

\begin{enumerate}
  \item 
Develop a research community that shares ideas and collaborates on 
research related to the evaluation and usability of languages and tools.
\item
Encourage the languages and tools communities to think more critically
about how usability affects the  design and
adoption of languages and tools.
\end{enumerate}


\section{Organizers}

The logistical organisation of this workshop will be done by the following organizers.

\begin{itemize}
\item \textbf{Emerson Murphy-Hill} is currently a
  postdoctoral fellow at the University of British Columbia in the
  Software Practices Lab with Gail Murphy, researching how 
  software developers find and adopt software tools.
  He recieved is Ph.D. from Portland State University in 2009.
  His research interests include human-computer interaction and software tools. 

\item \textbf{Shane Markstrum} recently defended his PhD in the
  Department of Computer Science, University of California, Los
  Angeles, USA. His PhD dissertation is titled \emph{Enforcing and
    Validating User-Extensible Type Systems} and was supervised by
  Todd Millstein. Shane has extensive experience in building
  domain-specific languages for type systems and building plugins for
  Eclipse that focus on language-oriented features. Shane has recently
  joined the faculty of Bucknell University as an assistant professor
  of Computer Science.  
  
\item \textbf{Craig Anslow} is a PhD student in the School of
  Engineering and Computer Science, Victoria University of Wellington,
  New Zealand. His PhD topic is \emph{Visual Software Analytics for
   Multi-Touch Tablesy} and is supervised by James Noble and
  Stuart Marshall. Craig has experience in building applications to
  support the evaluation of programming languages using information
  visualization techniques.

\end{itemize}

\section{Program Committee}

Along with the organizers, the following people will form the Program
Committee (PC) for the workshop and will help promote the workshop in
the programming languages and human-computer interaction communities.

\begin{itemize}

% Not confirmed yet
% \item \textbf{Andrew P. Black} is a Professor of Computer Science,
%   Portland State University, USA. His research interests are in the
%   area of programming languages, operating systems, object-oriented
%   systems and distributed computing, and more specifically in the
%   region where they overlap (such as language design for distributed
%   object-oriented computing). Andrew has been on the PC of OOPSLA
%   ('08, '05) and Chair of ECOOP ('05) and PC member ('07, '06, '05,
%   '03).


%TODO: just finished phd
\item \textbf{Donna Malayeri} is a PhD student in the Computer Science
  Department, Carnegie Mellon University, USA. Her PhD topic is
  \emph{Retroactive Abstraction, Extensibility and Inheritance in
    Object-Oriented Languages} and is supervised by Jonathan
  Aldrich. Donna's research interests include programming languages,
  including usability, tools, and environments; static analysis; and
  human-computer interaction.

\item \textbf{James Noble} is a Professor of Computer Science and
  Software Engineering within the School of Engineering and Computer
  Science, Victoria University of Wellington, New Zealand.  His
  research areas include Software Design, Programming Languages,
  Design Patterns; Human-Computer Interaction; Software Visualisation
  and Visual Languages; and the philosophy of Computer Science and
  Software Engineering. James has been on the PC of OOPSLA ('09, '06,
  '05, '02, '01) and ECOOP ('09, '08, '07, '05, '01).

\item \textbf{Ewan Tempero} is an Associate Professor in the
  Department of Computer Science, University of Auckland, New
  Zealand. The main goal of his research is to make programmers more
  productive, that is, help the people who actually produce code to do
  so faster, with less effort, fewer errors, and with more enjoyment
  than currently. His research interests are in the area of
  object-oriented programming languages, software reuse, tool support
  for programmers, software visualization, and the software
  development process.

\item \textbf{Christoph Treude} is

\item \textbf{Ben Widermann} is

\end{itemize}

\section{Anticipated Attendance}

We anticipate the following number of attendees:

\begin{itemize}
\item Minimum: 10 
\item Ideal: 30
\item Maximum: 60 
\end{itemize}

\section{Advertisement}

We will advertise this workshop by inviting participants of workshops
in the areas of language design, tools, and general usability
directly; as well as by emailing related mailing lists, posting on
blogs (where we were well-publicised last
year\footnote{http://pyre.third-bit.com/blog/archives/3113.html
http://wadler.blogspot.com/2009/07/evaluation-and-usability-of-programming.html
http://lambda-the-ultimate.org/node/3529}
contacting specific people known to be working in this area directly,
and through a mailing list that we established last year. In
addition we will maintain a website for presenting position papers and organizational
information.

\section{Participant Preparation}

Workshop participants should submit a position paper prior to one
month before the workshop. Position papers will be made available
through the workshop website and participants are encouraged to have
read the position papers before attending the workshop. Participants
are also asked to prepare a presentation to support their position
paper. We will accept papers (from 4 to 6 pages) that describe
work-in-progress or recently completed work based on the themes and
goals of the workshop or related topics, report on experiences gained,
question accepted wisdom, raise challenging open problems, or propose
speculative new approaches.

\section{Activities and Format}

This workshop will be run as a full-day workshop at Onward! 2010.  We
will have an introduction and keynote session in the morning followed
by the presentation of workshop papers.  The last session of the day
will include some workshop papers followed by a panel. 
If we have time we will prepare a poster for the
OOPSLA Welcome Reception.  Table~\ref{tab:schedule} outlines the
schedule of the format of the workshop.


\begin{table} [!htbp] % this puts tables exactly where you want them
\arrayrulecolor[gray]{0.8} %line color
\begin{tabularx}{\columnwidth}{l|X}
\textbf{Time}   & \textbf{Activity} \\
\hline
0830--0900    & Introductions \vspace{1mm} \\
0900--1000    & Key Note (speaker to be determined) \vspace{1mm} \\
Break              & ~\vspace{1mm}\\
1030--1200   & Presentation of workshop papers \vspace{1mm}\\
Lunch             & ~\vspace{1mm}\\
1300--1430   & Presentation of workshop papers \vspace{1mm}\\
Break              & ~\vspace{1mm}\\
1500--1600    & Presentation of workshop papers \vspace{1mm}\\
1600--1700    & Panel (presenters to be determined) \\

\end{tabularx}
\caption{Workshop Schedule}
\label{tab:schedule}
\end{table}

\section{Post-workshop Activities}

We will publish our participant's papers as a technical report.
We aim to continue hosting this workshop in future years.

\section{Special Requirements}

We require the room to have Internet access, power strips for
attendees to plug their laptops into, and a computer projector and
screen for attendees to present their presentations of their position
papers.

\end{document}
