\documentclass{acm_proc_article-sp}

\usepackage{ifthen}

\newboolean{hidecomments}
\setboolean{hidecomments}{false}
\ifthenelse{\boolean{hidecomments}}
{\newcommand{\nb}[2]{}}
{\newcommand{\nb}[2]{
    \fbox{\bfseries\sffamily\scriptsize#1}
    {\sf\small$\blacktriangleright$
      {#2} $\blacktriangleleft$}}}
\newcommand\apb[1]{\nb{APB}{#1}}
\newcommand\emh[1]{\nb{EMH}{#1}}


\begin{document}

\title{Evaluation and Usability of Programming Languages and Tools (PLATEAU)}

\numberofauthors{3}
\author{
\alignauthor Craig Anslow\\
       \affaddr{School of Engineering and Computer Science}\\
       \affaddr{Victoria University of Wellington}\\
       \affaddr{Wellington, New Zealand}\\
       \email{craig@ecs.vuw.ac.nz}
\alignauthor Shane Markstrum\\
       \affaddr{Computer Science Department}\\
       \affaddr{University of California, Los Angeles}\\
       \affaddr{California, USA}\\
       \email{smarkstr@cs.ucla.edu}
\alignauthor Emerson Murphy-Hill\\
       \affaddr{Department of Computer Science}\\
       \affaddr{Portland State University}\\
       \affaddr{Oregon, USA}\\
       \email{emerson@cs.pdx.edu}
       }

\date{}

\maketitle
\begin{abstract}

  \emh{Andrew Black suggested that we go even further in linking
    usability and languages/tools.  I've revised the abstract a bit to
    that end.  (These comments can be turned off via
    hidecomments=true) } Programming languages exist to enable
  programmers to develop software effectively.  But how
  \emph{efficiently} programmers can write software depends on the
  usability of the languages and tools that they develop with.  The
  aim of this workshop is to discuss methods, metrics and techniques
  for evaluating the usability of languages and language tools.  The
  supposed benefits of such languages and tools cover a large space,
  including making programs easier to read, write, and maintain;
  allowing programmers to write more flexible and powerful programs;
  and restricting programs to make them more safe and secure. We plan
  to gather the intersection of researchers in the object-oriented
  programming language and human computer interaction communities who
  are working on languages and tools to share their research and
  discuss the future of usability and evaluation of programming
  languages. We are also interested in the input of other members of
  the programming language community working on related areas, such as
  refactoring, design patterns, program analysis, program
  comprehension, software visualization, end-user programming, and
  aspect-oriented software development.

\end{abstract}

\category{D.3.2}{Programming Languages}{Languages Classifications}[Object-oriented languages]

\terms{Design}

\keywords{Evaluation, Object-Oriented Programming Languages, Tools, Usability} 

\section{Main Themes and Goals}

At the Programming Languages Grand Challenges panel at POPL 2009, Greg
Morrisett claimed that one of the great neglected areas in programming
language research is research that bridges the gap between PL and HCI:
evaluation of the usability of programming languages and tools. This
is evident by looking at the recent programs of major languages
conferences such as POPL, PLDI, OOPSLA, and ECOOP. The object-oriented
conferences tend to have at least one or two papers in the areas of
corpus analysis or evaluation methodologies, but the authors of the
papers seem to avoid using the results of their studies to make
conclusions about the languages or tools themselves. Software
engineering and HCI conferences tend to have slightly better support
of language usability analysis (CHI 2009 has three tracks that push in
this direction) but have limited visibility to the language community.

This workshop aims to be a first step in filling that void by
developing and stimulating discussion of usability and evaluation of
programming languages and tools with respect to language design and
related areas.  We will consider: empirical studies of programming
languages; methodologies and philosophies behind language and tool
evaluation; software design metrics and their relations to the
underlying language; user studies of language features and software
engineering tools; visual techniques for understanding programming
languages; critical comparisons of programming paradigms, such as
object-oriented vs. functional; and tools to support evaluating
programming languages.  Our goals are to develop a community of domain
experts to share ideas and collaborate on research related to the
evaluation and usability of languages and tools and to get the
languages and tools communities to begin thinking more critically
about how these concerns should affect language and tool design and
proliferation.



\section{Organizers}

The logistical organisation of this workshop will be done by the following organizers.

\begin{itemize}
\item \textbf{Craig Anslow} is a PhD student in the School of
Engineering and Computer Science, Victoria University of Wellington,
New Zealand. His PhD topic is Visual Software Analytics of Object
Oriented Software Using Multi-touch Visualization Displays and
supervised by James Noble and Stuart Marshall. Craig has participated
in several workshops at OOPSLA and CHI as well as regional conferences
and is looking forward to the challenge of organising a workshop this
year.

\item \textbf{Shane Markstrum} is a PhD student in the Department of
Computer Science, University of California, Los Angeles, USA. His PhD
topic is Enforcing and Validating User-Extensible Type Systems and is
supervised by Todd Millstein. Shane has extensive experience in
building domain-specific languages for type systems and building
plugins for Eclipse that focus on language-oriented features.

\item \textbf{Emerson Murphy-Hill} recently defended his PhD in the
Department of Computer Science, Portland State University, USA. His
dissertation was titled \emph{Programmer Friendly Refactoring Tools} and was supervised by
Andrew Black. Emerson will begin a postdoctoral fellowship on April 1st, 2009 with Gail Murphy at
University of British Columbia in the Software Practices Lab.
\end{itemize}

\section{Program Committee}

Along with the organizers, the following people will form the Program Committee (PC) for
the workshop and will help promote the workshop in the programming
language community.

\begin{itemize}

\item \textbf{Andrew Black} is a Professor of Computer Science, Portland State University, USA. His research interests are in the area of programming languages, operating systems, object-oriented systems and distributed computing, and more specifically in the region where they overlap (such as language design for distributed object-oriented computing). Andrew has been on the PC of OOPSLA ('08, '05) and Chair of ECOOP ('05) and PC member ('07, '06, '05, '03).

\item \textbf{Larry Constantine} - is a Professor in the Department of Mathematics and
Engineering at the University of Madeira where he teaches in the dual-degree
program that he helped organize with Carnegie-Mellon University in the
United States. He heads the Laboratory for Usage-centered Software
Engineering, a research and development group dedicated to making technology
more useful and usable. He is an award-winning designer and author,
recipient of the 2009 Stevens Award for his contributions to design and
design methods, and a Fellow of the Association for Computing Machinery.

\item \textbf{Jeff Foster} is an Assistant Professor in the Computer Science Department and UMIACS at the University of Maryland, College Park. The goal of his research is to develop practical tools and techniques to improve software quality. His research interests are programming languages, software engineering, advanced static type systems, scalable constraint-based analysis, and building tools that implement his ideas.

\item \textbf{Bob Fuhrer} is a Research Staff Member at the IBM T.J. Watson Research Center in Hawthorne, NY, USA. He currently works in the Program Analysis and Transformation group on advanced refactoring and static analysis. Robert leads the SAFARI project, whose goal is to develop a meta-tooling platform for creating full-featured language-specific IDEs for Eclipse, including editing, navigation, analysis and refactoring. Robert is also a member of the team creating the X10 programming language.

\item \textbf{Donna Malayeri} is a PhD student in the Computer Science Department, Carnegie Mellon University, USA. Her PhD topic is Retroactive Abstraction, Extensibility and Inheritance in Object-Oriented Languages and is supervised by Jonathan Aldrich. Donna's research interests include programming languages, including usability, tools, and environments; static analysis;  and human-computer interaction. 

\item \textbf{Stuart Marshall} is a Lecturer in the School of Engineering and Computer Science, Victoria University of Wellington, New Zealand. His research interests are in the area of mobile user interfaces, software reuse, and software visualization.

\item \textbf{Todd Millstein} is an Assistant Professor in the Department of Computer Science, University of California, Los Angeles, USA. His research areas include ... . Todd has been on the PC of OOPSLA ('09) and ECOOP ('08).

\item \textbf{James Noble} is a Professor of Computer Science (Software Engineering) within the School of 
Engineering and Computer Science, Victoria University of Wellington, New Zealand. 
His research areas include Software Design, Programming Languages, Design Patterns; 
Human Computer Interaction; Software Visualisation and Visual Languages; and the philosophy of Computer Science and Software Engineering. James has been on the PC of OOPSLA ('09, '06, '05, '02, '01) and ECOOP ('09, '08, '07, '05, '01). James has been chair of DLS ('09), WikiSym ('06), and KoalaPLoP ('02). James has also been on the PC of a range of conferences including POPL ('07), AOSD ('08), WikiSym ('08, '07, '06, '05), VL/HCC ('08, '07, '06), forUSE ('02), and TOOLS Pacific ('02). Finally, James is also a director of the Hillside Group and Foundation Editor-In-Chief of Transactions on Pattern Languages of Programming, ('07-present).

\item \textbf{Ewan Tempero} is an Associate Professor in the Department of Computer Science, University of Auckland, New Zealand. The main goal of his research is to make programmers more productive, that is, help the people who actually produce code to do so faster, with less effort, fewer errors, and with more enjoyment than currently. His research interests are in the area of object-oriented programming languages, software reuse, tool support for programmers, software visualization, and the software development process.

\end{itemize}

\section{Anticipated Attendance}

We anticipate the following number of attendees:

\begin{itemize}
\item Minimum: 10 
\item Ideal: 25 
\item Maximum: 40 
\end{itemize}

\section{Advertisement}

We will advertise this workshop by inviting participants of workshops
in the areas of language design, tools, and general usability
directly; as well as by emailing related mailing lists, posting on
blogs such as Lambda the Ultimate, and contacting specific people
known to be working in this area directly. In addition we will
maintain a website for presenting position papers and organizational
information.

\section{Participant Preparation}

Workshop participants should submit a position paper prior to one
month before the workshop. Position papers will be made available
through the workshop website and participants are encouraged to have
read the position papers before attending the workshop. Participants
are also asked to prepare a presentation to support their position
paper. We will accept papers (from 4 to 6 pages) that describes
work-in-progress or recently completed work based on the themes and
goals of the workshop or related topics, report on experiences gained,
question accepted wisdom, raise challenging open problems, or propose
speculative new approaches.

\section{Activities and Format}

This workshop will be run as a full-day workshop at OOPSLA 2009. 
We will have an introduction and keynote session in the morning followed 
by the presentation of workshop papers. 
The last session of the day will include some workshop papers followed by 
a panel on the ``Gap Between Programming Languages and Human Computer 
Interaction''.
If we have time we will prepare a poster for the Welcome Reception. 
\emh{uh? when is the welcome reception?
Have a poster session isn't a bad idea.
For instance, in the event that we have more than 10 accepted papers, 
the ``weaker'' papers in excess of 10 could present their as a poster during some 1-hour session}
Table \ref{tab:schedule} outlines the schedule of the format of the workshop.


\begin{table} [!htbp] % this puts tables exactly where you want them
\caption{Workshop Schedule}
\begin{tabular}{|l|l|}
\hline \textbf{Time}   & \textbf{Activity}\\
\hline 0830--0900    & Introductions\\
\hline 0900--1000    & Key Note presentation by\\
                                    & Larry Constantine\\
                                    & "Usability in a Post-Waterfall World"\\
\hline Break              & \\
\hline 1030--1200   & Presentation of workshop papers\\
\hline Lunch             & \\
\hline 1300--1430   & Presentation of workshop papers\\
\hline Break              & \\
\hline 1500--1600    & Presentation of workshop papers\\
\hline 1600--1700    & Panel: Bridging The Gap Between\\
                     & Programming Languages and\\
                     & Human Computer Interaction\\
\hline
\end{tabular}
\label{tab:schedule}
\end{table}

\section{Post-workshop Activities}

We will establish a mailing list and wiki for participants 
to continue discussion and develop a community of 
interested parties. 
We will aim to host a similar workshop in future years.

\section{Special Requirements}

We require the room to have Internet access, power strips for 
attendees to plug their laptops into, and a computer projector 
and screen for attendees to present their presentations.

% \bibliographystyle{abbrv}
% \bibliography{sigproc} 

\end{document}
