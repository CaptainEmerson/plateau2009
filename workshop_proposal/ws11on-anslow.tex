\documentclass{sigplanconf}

\usepackage{ifthen}
\usepackage{colortbl}
\usepackage{tabularx}

\newboolean{hidecomments}
\setboolean{hidecomments}{true}
\ifthenelse{\boolean{hidecomments}}
{\newcommand{\nb}[2]{}}
{\newcommand{\nb}[2]{
    \fbox{\bfseries\sffamily\scriptsize#1}
    {\sf\small$\blacktriangleright$
      {#2} $\blacktriangleleft$}}}
\newcommand\apb[1]{\nb{APB}{#1}}
\newcommand\emh[1]{\nb{EMH}{#1}}


\begin{document}

\authorpermission
\conferenceinfo{SPLASH'11 Companion,} {October 22--27, 2011, Portland, Oregon, USA.}
\CopyrightYear{2011}
\copyrightdata{978-1-4503-0940-0/11/10} 

\title{Evaluation and Usability of Programming Languages and Tools (PLATEAU)}

\authorinfo{Craig Anslow}
           {Victoria University of Wellington}
           {craig@ecs.vuw.ac.nz}
     \authorinfo{Shane Markstrum}
           {Google}
           {smarkstr@google.com}      
\authorinfo{Emerson Murphy-Hill}
           {North Carolina State University}
           {emerson@csc.ncsu.edu}
\date{}

\maketitle
\begin{abstract}

  Programming languages exist to enable programmers to develop
  software effectively.  But how \emph{efficiently} programmers can
  write software depends on the usability of the languages and tools
  that they develop with.  The aim of this workshop is to discuss
  methods, metrics and techniques for evaluating the usability of
  languages and language tools.  The supposed benefits of such
  languages and tools cover a large space, including making programs
  easier to read, write, and maintain; allowing programmers to write
  more flexible and powerful programs; and restricting programs to
  make them more safe and secure. We plan to gather the intersection
  of researchers in the programming language, programming tool, and
  human-computer interaction communities to share their research and
  discuss the future of evaluation and usability of programming
  languages and tools. We are also interested in the input of other
  members of the programming research community working on related
  areas, such as refactoring, design patterns, program analysis,
  program comprehension, software visualization, end-user programming,
  and other programming language paradigms.

\end{abstract}

\category{D.3.0}{Programming Languages}{Standards}
\category{H.1.2}{User/Machine Systems}{Human Factors}


\terms{Human Factors, Languages}

\keywords{Evaluation, Programming Languages, Tools, Usability} 

\section{Main Themes and Goals}

At the Programming Languages Grand Challenges panel at POPL 2009, Greg
Morrisett claimed that one of the great neglected areas in programming
languages research is the bridge between programming languages and
human-computer interaction: the evaluation of the usability of
programming languages and tools. This is evident by the recent
research programs of major languages conferences such as POPL, PLDI,
OOPSLA, and ECOOP. The object-oriented conferences tend to have at
least one or two papers in the areas of corpus analysis or evaluation
methodologies, but the authors of the papers seem to avoid using the
results of their studies to make conclusions about the languages or
tools themselves. Software engineering and human-computer interaction
conferences tend to have better support of language usability analysis
(CHI 2009 had three tracks that showcase research in this direction),
but have limited visibility to the programming languages community.

Following on from our previous workshops at OOPSLA/Onward! 
2009 and SPLASH 2010, this workshop aims to fill that void by
developing and stimulating discussion of usability and evaluation of
programming languages and tools with respect to language design and
related areas. We will consider: empirical studies of programming
languages; methodologies and philosophies behind language and tool
evaluation; software design metrics and their relations to the
underlying language; user studies of language features and software
engineering tools; visual techniques for understanding programming
languages; critical comparisons of programming paradigms, such as
object-oriented vs. functional; and tools to support evaluating
programming languages. We have two goals:

\begin{enumerate}
  \item 
Develop a research community that shares ideas and collaborates on 
research related to the evaluation and usability of languages and tools.
\item
Encourage the languages and tools communities to think more critically
about how usability affects the  design and adoption of languages and tools.
\end{enumerate}


\section{Organizers}

\begin{itemize}
\item \textbf{Craig Anslow} is currently a PhD student in the School of
  Engineering and Computer Science, Victoria University of Wellington,
  New Zealand. His PhD topic is \emph{Multi-touch Table User
    Interfaces for Collaborative Software Visualization} and is
  supervised by James Noble and Stuart Marshall. His research interests include the evaluation and usability of programming languages and software, software visualization, and multi-touch user intefaces.

\item \textbf{Shane Markstrum} is currently a Software Engineer at Google in New York, USA. Prior to joining Google he was an Assistant Professor in the Computer Science
  department at Bucknell University. He received his Ph.D. from the University of
  California, Los Angeles in 2009. His research interests include
  domain-specific languages and tools for extensible type systems; and 
  building tool support for non-traditional language constructs.
  
\item \textbf{Emerson Murphy-Hill} is currently an Assistant Professor at North Carolina State University, USA. Prior to joining the NCSU faculy he was a postdoctoral researcher at the University of British Columbia in the Software Practices Lab with Gail Murphy. 
  He recieved is Ph.D. from Portland State University in 2009.
  His research interests include human-computer interaction and software tools. 
\end{itemize}

\section{Program Committee}

The following people will form the Program Committee (PC) for the workshop and will help promote the workshop in the programming languages and human-computer interaction communities.

\begin{itemize}
\item  Craig Anslow - Victoria University of Wellington, New Zealand
\item  Rob DeLine - Microsoft Research, USA
\item Jeff Carver - University of Alabama, USA
\item Jonathan Edwards - MIT, USA
\item Matthias Hauswirth - University of Lugano, Switzerland
\item Donna Malayeri - Microsoft, USA
\item Shane Markstrum - Google, USA
\item  Emerson Murphy-Hill - North Carolina State University, USA
\item James Noble - Victoria University of Wellington, New Zealand
\item Portia O'Callaghan - MathWorks, USA
\item Marian Petre - The Open University, England
\item Caitlin Sadowski - University of California Santa Cruz, USA
\item  Alessandro Warth - Viewpoints Research Institute, USA 
\end{itemize}


\section{Anticipated Attendance}

We anticipate the following number of attendees:

\begin{itemize}
\item Minimum: 10 
\item Ideal: 25
\item Maximum: 40 
\end{itemize}

\section{Advertisement}

We will advertise this workshop by inviting participants of workshops
in the areas of language design, tools, and general usability
directly; as well as by emailing related mailing lists, posting on
blogs contacting specific people known to be working in this area directly,
and through a mailing list our group mailing list. In addition we will maintain a website\footnote{http://ecs.victoria.ac.nz/Events/PLATEAU/} for presenting position papers and organizational information.

\section{Participant Preparation}\label{preparation}

Workshop participants should submit a paper prior to one
month before the workshop. Papers will be made available
through the workshop website and participants are encouraged to have
read the papers before attending the workshop. Participants
are also asked to prepare a presentation to support their position
paper. We will accept two tracks of papers: short-form (up to 4 pages) and long-form (up to 10 pages). 

\vfill\eject

In both tracks, we will look for papers that describe work-in-progress or recently completed work based on the themes and goals of the workshop or related topics, report on experiences gained, question accepted wisdom, raise challenging open problems, or propose
speculative new approaches. We plan to have the workshop proceedings published in the ACM Digital Library, where short-form papers will be represented by abstract only and long-form papers will be published in full.

\section{Activities and Format}

This workshop will be run as a full-day workshop at SPLASH and Onward! 2011.  We
will have an introduction and keynote session in the morning followed
by the presentation of workshop papers in three separate sessions. We will prepare a poster for the SPLASH/Onward! Welcome Reception representing the presentations of the papers. Table~\ref{tab:schedule} outlines the schedule of the format of the workshop.


\begin{table} [!htbp] % this puts tables exactly where you want them
\arrayrulecolor[gray]{0.8} %line color
\begin{tabularx}{\columnwidth}{l|X}
\textbf{Time}   & \textbf{Activity} \\
\hline
0830--0900    & Introductions \vspace{1mm} \\
0900--1000    & Key Note Presentation: Brad Myers \vspace{1mm} \\
                       &``Inherent vs. Accidental vs. Intentional Difficulties in Programming"\\
1000--1030     & Morning Break~\vspace{1mm}\\
1030--1200   & Presentation of workshop papers \vspace{1mm}\\
1200-1330     & Lunch Break~\vspace{1mm}\\
1330--1500   & Presentation of workshop papers \vspace{1mm}\\
1500--1530    & Afternoon Break~\vspace{1mm}\\
1530--1700    & Presentation of workshop papers \vspace{1mm}\\
1700--1730    & Participant Feedback and Organizers Report  \\
\end{tabularx}
\caption{Workshop Schedule}
\label{tab:schedule}
\end{table}

\section{Post-workshop Activities}

We will publish our participant's papers in the ACM Digital Library, as outlined earlier (\S \ref{preparation}). We aim to continue hosting this workshop in future years.

\section{Special Requirements}

We require the room to have Internet access, power strips for
attendees to plug their laptops into, and a computer projector and
screen for attendees to show the presentations of their position
papers.

\end{document}
