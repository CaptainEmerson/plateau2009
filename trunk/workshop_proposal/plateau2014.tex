\documentclass{sigplanconf}

\usepackage[T1]{fontenc}
\usepackage{ifthen}
\usepackage{colortbl}
\usepackage{tabularx}
%\usepackage{flushend}
\usepackage{url}

\newboolean{hidecomments}
\setboolean{hidecomments}{true}
\ifthenelse{\boolean{hidecomments}}
{\newcommand{\nb}[2]{}}
{\newcommand{\nb}[2]{
    \fbox{\bfseries\sffamily\scriptsize#1}
    {\sf\small$\blacktriangleright$
      {#2} $\blacktriangleleft$}}}
\newcommand\apb[1]{\nb{APB}{#1}}
\newcommand\emh[1]{\nb{EMH}{#1}}


\begin{document}

\title{Evaluation and Usability of Programming Languages and Tools (PLATEAU)}

\authorinfo{Thomas LaToza}
           {University of California, Irvine, USA}
           {tlatoza@uci.edu} 
\authorinfo{Craig Anslow}
           {University of Calgary, Canada}
           {craig.anslow@ucalgary.ca}
\authorinfo{Joshua Sunshine}
           {Carnegie Mellon University, USA}
           {sunshine@cs.cmu.edu}
\date{}

\maketitle
\begin{abstract}

  Programming languages exist to enable programmers to develop
  software effectively.  But how \emph{efficiently} programmers can
  write software depends on the usability of the languages and tools
  that they develop with.  The aim of this workshop is to discuss
  methods, metrics and techniques for evaluating the usability of
  languages and language tools.  The supposed benefits of such
  languages and tools cover a large space, including making programs
  easier to read, write, and maintain; allowing programmers to write
  more flexible and powerful programs; and restricting programs to
  make them more safe and secure. We plan to gather the intersection
  of researchers in the programming language, programming tool, and
  human-computer interaction communities to share their research and
  discuss the future of evaluation and usability of programming
  languages and tools. We are also interested in the input of other
  members of the programming research community working on related
  areas, such as refactoring, design patterns, program analysis,
  program comprehension, software visualization, end-user programming,
  and other programming language paradigms.

\end{abstract}

\category{D.3.0}{Programming Languages}{Standards}
\category{H.1.2}{User/Machine Systems}{Human Factors}


\terms{Human Factors, Languages}

\keywords{Evaluation, Programming Languages, Tools, Usability} 

\section{Main Themes and Goals}

Following on from the four previous workshops at SPLASH, this workshop aims to bring together practitioners and
researchers interested discussing usability and evaluation of
programming languages and tools with respect to language design and
related areas. We will consider: empirical studies of programming
languages; methodologies and philosophies behind language and tool
evaluation; software design metrics and their relations to the
underlying language; user studies of language features and software
engineering tools; visual techniques for understanding programming
languages; critical comparisons of programming paradigms, such as
object-oriented vs. functional; and tools to support evaluating
programming languages. We have two goals:

\begin{enumerate}
  \item 
Develop and sustain a research community that shares ideas and collaborates on 
research related to the evaluation and usability of languages and tools.
\item
Encourage the languages and tools communities to think more critically
about how usability affects the  design and adoption of languages and tools.
\end{enumerate}


\section{Organizers}

\begin{itemize}
  
\item \textbf{Thomas LaToza} is currently a Postdoctoral Research Scholar at the University of California, Irvine, USA.  He received his PhD from Carnegie Mellon University in 2012 and was advised by Brad A. Myers and Jonathan Aldrich. His research interests are software engineering, with a focus on human aspects of software development. 

 \item \textbf{Craig Anslow} is currently a Postdoctoral Research Scholar at the University of Calgary, Canada. He received his PhD from Victoria University of Wellington, New Zealand in 2013 and was advised by James Noble and Stuart Marshall. His research interests include the evaluation and usability of programming languages and tools, and applying information visualization to software engineering.
  
\item \textbf{Joshua Sunshine} is currently a Systems Scientist in the Institute for Software Research at Carnegie Mellon University.  He received his PhD from Carnegie Mellon University in 2013 and was advised by Jonathan Aldrich. His research interests are at the intersection of programming languages and software engineering. He is particularly interested in better understanding of the factors that influence the usability of reusable software components.
  
\end{itemize}

\section{Program Committee}

The Program Committee (PC) for the workshop will review papers and help promote the workshop in the programming languages, software engineering, and human-computer interaction communities. The following people have agreed to serve on the PC:

\begin{itemize}
\item Andew Begel - Microsoft Research, USA
\item Andrew Black - Portland State University, USA
\item Alan Blackwell - Cambridge University, UK
\item Jeff Carver - University of Alabama, USA
%\item Yvonne Coady - University of Victoria, Canada
\item Sebastian Erdweg - TU Darmstadt, Germany
\item Thomas Fritz - University of Zurich, Switzerland
%\item Ronald Garcia - University of British Columbia, Canada
\item Stefan Hanenberg - University of Duisburg-Essen, Germany
%\item Matthias Hauswirth - University of Lugano, Switzerland
\item Ciera Jaspan - Google, USA
\item Caitlin Kelleher - Washington University St Louis, USA
\item Leo Meyerovich - University of California, Berkeley, USA
\item Gail Murphy - University of British Columbia, Canada
\item Emerson Murphy-Hill - North Carolina State University, USA
\item James Noble - Victoria University of Wellington, New Zealand
\item Chris Parnin - Georgia Institute of Technology, USA
%\item Caitlin Sadowski - Google, USA
\item Janet Siegmund - University of Passau, Germany
\item Andreas Stefik - University of Nevada, USA
\item Chris Scaffidi - Oregon State University, Corvallis, USA
\item \'Eric Tanter - University of Chile, Chile
\end{itemize}

\section{Anticipated Attendance}

We anticipate the following number of attendees:

\begin{itemize}
\item Minimum: 10 
\item Ideal: 20
\item Maximum: 30 
\end{itemize}

\section{Participant Preparation}\label{preparation}

Authors should submit a paper for review approximately two months before the workshop. Papers will be reviewed and discussed by the PC quickly and authors will be notified before the SPLASH early registration deadline. Papers will be made available in advance of the workshop through the workshop website and all participants will be encouraged to read the the papers before attending the workshop. Authors will prepare a short presentation to support their paper. To help improve the quality of submissions, the call for papers will link to exemplary papers from previous conferences.

We will solicit three kinds of papers: research papers,  position papers, and hypothesis papers. Research papers can  describe work-in-progress or recently completed work on the themes and goals of the workshop or related topics. Position papers can report on experience, question accepted wisdom, raise challenging open problems, or propose speculative new approaches. Hypothesis papers will identify and collect the unsubstantiated beliefs of the research community or software industry. The hypotheses can be collected from mailing lists, blog posts, paper introductions,  developer forums, or interviews. Authors will be encouraged to document the source or sources of each hypothesis and to discuss the impact of the hypotheses on research or practice. Research papers will be limited to 10 pages in length and both position and hypothesis papers will be limited to 2 pages in length. Papers will be published in the ACM Digital Library at the authors' discretion.

In addition, we will invite the authors of papers of topical papers focus area presented at major programming languages, software engineering, or human-computer interaction conferences to present posters or ``lightning presentations'' (less than 5 minute) at the workshop. This idea has been used successfully for many years at the Symposium on Usable Privacy and Security (SOUPS), which is a similarly interdisciplinary venue. The organizers are in the process of combing the 2013 and 2014 proceedings of CHI, CSE, FSE, OOPSLA, ICFP, POPL, PLDI, ECOOP, CSCW, VLHCC, and PPIG for topical papers. We are also considering presenting a PLATEAU award to one of these same papers.

\section{Keynote}

\section{Activities and Format}
\begin{table} [!t] % this puts tables exactly where you want them
\arrayrulecolor[gray]{0.8} %line color
\begin{tabularx}{\columnwidth}{|l|X|}
\hline
\textbf{Time}   & \textbf{Activity} \\
\hline
0830--0900    & Introductions \vspace{1mm} \\
0900--1000    & Keynote Presentation \vspace{1mm} \\
1000--1030    & Morning Break~\vspace{1mm}\\
1030--1200    & Presentation of position / research papers \vspace{1mm}\\
1200--1330    & Lunch Break~\vspace{1mm}\\
1330--1445    & Posters / lightning talks of research presented at other conferences\vspace{1mm}\\
1445--1515    & Afternoon Break~\vspace{1mm}\\
1515--1715    & Hypotheses breakout session \vspace{1mm}\\
1715--1730    & Participant Feedback and Organizers Report  \\
\end{tabularx}
\caption{Workshop Schedule}
\label{tab:schedule}
\end{table}
This workshop will be run as a full-day mini-conferene.  After the introductory remarks, Josh Bloch, designer of much of the Java Standard Library and popular technology author, will deliver the keynote. Dr. Block's expertise in API design is very relevant to PLATEAU. Research and position papers will be presented in the second morning session. After lunch, related work first presented at other venues will be repeated as posters or lightning presentations. Finally, we will conduct breakout sessions in which we will collate, consolidate, and discuss the hypotheses presented in hypothesis papers. Finally, we will wrap up with participant feedback and reports from the organizers.. Table~\ref{tab:schedule} outlines the schedule of the workshop.


\section{Advertisement and Publicity}

Last years edition of PLATEAU was canceled because insufficient papers were submitted to fill the program. There are five factors that are different this year, which we believe will prevent a repeat of the cancellation:
\begin{enumerate}
\item We have attracted a keynote speaker, Josh Bloch, with a wide following in both academia and industry. 
\item We will publicize the CFP and the workshop more aggressively, especially by encouraging our PC to spur contributions. 
\item We have de-emphasized research papers in the program and are therefore less reliant on them for success. 
\item The invitations to authors of related work first presented at other venues should help reach new participants. 
\item The organizing committee has turned over completely, injecting new energy into promotion.
\end{enumerate}

We will advertise this workshop by inviting participants of workshops
in the areas of language design, tools, and general usability
directly; as well as by emailing related mailing lists, posting on
blogs contacting specific people known to be working in this area directly,
and through our group mailing list. The workshop organizers will design and build a website to host conference materials as we have in previous years~\cite{website-sites,website-vuw}.

\section{Workshop Requirements}

We require the room to have easels, pens, and paper for discussion; 
Internet access; power strips for attendees to plug their laptops into; and a computer projector and
screen for attendees to show the presentations of their
papers.


\section{Post-workshop Activities}

We will publish our participant's papers in the ACM Digital Library, as outlined earlier (\S \ref{preparation}). Participants of the workshop will be encouraged to subscribe to our mailing list to continue the workshop discussion. We aim to continue hosting this workshop in future years.

\begin{thebibliography}{1}
\bibitem{website-sites}
 \url{https://sites.google.com/site/workshopplateau/}
 \bibitem{website-vuw}
\url{http://ecs.victoria.ac.nz/Events/PLATEAU/WebHome}
\end{thebibliography}

\end{document}
